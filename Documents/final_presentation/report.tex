\documentclass[12pt]{article}

\usepackage{amsfonts,amssymb,amsmath,amsthm}
\usepackage{stmaryrd} %for special brackets
\usepackage{lipsum}
\usepackage{graphicx}
\graphicspath{Images}
\usepackage[hidelinks]{hyperref}
\usepackage{array}
\usepackage{bm} %for bold in math equations
\usepackage{float} %to force the position of some figures
\usepackage{biblatex}
\usepackage[colorinlistoftodos]{todonotes}

\title{Report}
\author{Lucas SALAND}
\date{\today}

\begin{document}
\section*{Introduction}

\section{AI image compression}
\subsection{Autoencoder and latent space}
The idea behind AI image compression is to have a compact representation of images using deep learning. To do so, we can use an autoencoder. The encoder part is used to obtain a representation of images in latent space. The coefficients of this representation is then quantized. We can then reconstruct a quantized image by feeding the quantized latent representation to the decoder part of the autoencoder. The training is done by optimizing the weighted sum of the rate and distortion.
\begin{figure}[H]
    \centering
    \includegraphics*[width=.6\textwidth]{./img/transform_coding.png}
    \caption[short]{$x$: uncompressed image, $y$: latent representation of $x$, $\hat{y}$: quantized latent representation of $x$, $\hat{x}$ reconstructed image}
\end{figure}
Compressed images obtained with the autoencoder approach are smoother than the ones obtained with JPEG.
\begin{figure}
    \centering
    \includegraphics[width=\textwidth]{./img/jpeg_ai.png}
    \caption[short]{AI compression vs JPEG compression (End To End Image Compression paper)}
\end{figure}

A pytorch implementation of this idea was provided by the iclr\_17\_compression repository available on Github. Nevertheless, no model checkpoint nor training data were provided.


\subsection{Training the model}
The generation of the training data was done using a script provided on the github repository. With this script, 80G of training data are generated. Once the training data has been generated, the training of the model can be started. The model used in all other experiments in this report has been trained over 1590000 steps across 6 epochs. The base models from which all other model in this project are derived from is called ImageCompressor.
\begin{figure}[H]
    \centering
    \includegraphics[width=\textwidth]{./img/ImageCompressor.png}
    \caption[short]{ImageCompressor model}
\end{figure}

\section{Steganography}
\subsection{Naive insertion method}
\subsubsection{Least significant bit modification}
The least significant bit modification consists in adding or substracting 1 to quantized coefficients of the latent representation with a given probability. We can model the modification as follows:

\begin{itemize}
    \item $c_i: \text{quantized coefficient in the latent space}$
    \item $\mathcal{E}
    _i \sim Categorical_{\{-1,0,1\}},\\
    \ \mathbb{P}(\mathcal{E}_i = -1) = \mathbb{P}(\mathcal{E}_i = 1) = p,\\
    \mathbb{P}(\mathcal{E}_i = 0) = 1-2p$ 
    \item $\hat{C}_i = c_i + \mathcal{E}_i \sim Categorical_{\{c_i-1,\ c_i,\ c_i+1\}}:$ quantized coefficient after modification
\end{itemize}

For each coeffcient $c_i$ we sample a realization $\epsilon_i$ of $\mathcal{E}_i$ and obtain the modified version of this coefficient $\hat{c_i}$ which is a realization of $\hat{C_i}$.\\
The image reconstructed from the original latent representation is called the \textbf{cover} whereas the one reconstructed from the modified latent representation is called the \textbf{stego}.

\begin{figure}[H]
    \centering
    \includegraphics[width=0.6\textwidth]{./img/naive_insertion.png}
    \caption[short]{diagram of the probabilities of modification of coefficients}
    \label{fig:probabilities}
\end{figure}
The insertion rate is given by $H_3(p) = -2p*log_2(p) - (1-2p)log_2(1-2p)$ in bits per coefficient. The size of the message we embbed is given by size = number coefficients in latent space * insertion rate.
\subsubsection{Implementation}
The insertion is done on the quantized coefficients in latent space. Thus, a new model ImageCompressorSteganography based on ImageCompressor contains the necessary steps to perform the insertion given a probability p (see \autoref{fig:probabilities}). The first implementation of the LSB modification was implemented using 3 for-loops. Due to low performances of python for-loops, a new implementation was done using pytorch's multinomial method, enabling computations to be performed 30 times faster by taking advantage of GPU's computational power.

\subsection{Side information: quantization error}
The LSB modification is a very basic approach to perform steganography. Its simplicity makes it easy to detect. Every coeffcient in the latent representation has the same probability of modifications. To make it less detectable, we can use the quantization error: the difference between original coefficients and quantized coefficients in latent space. The goal is to give higher probability of modification where the quantization error is important and smaller probabilities where the quantization error is close to zero.  
\subsubsection{Non-constant probability of modification}
To obtain different probabilities of modification for each coefficient, an optimization problem under constraint must be solved, the unknown is $\lambda$:
\begin{itemize}
    \item $\forall i \in \llbracket 1, \# coef \rrbracket,\ \forall k \in \{-1,\ 0,\ 1 \},\ p_{i,k} = \frac{e^{-\lambda c_{i,k}}}{e^{-\lambda c_{i,-1}} + e^{-\lambda c_{i,0}} + e^{-\lambda c_{i,1}}}$, probability to add k to the i-th coefficient
    \item $\displaystyle\sum_{i=1}^{\# coef}{H_3(p_{i,-1},p_{i,0},p_{i,1})} = M$, where M is the size of the message to embbed
\end{itemize}
This problem is solved using an iterative algorithm. The code for this algorithm was kindly provided by Patrick Bas.

\subsubsection{Implementation (and visual impact on image)}
\subsubsection{Modifying the cost}

\section{Steganalysis}
\subsection{JIN SRNet}
Steganography detector based on deep convolutional neural network, pretrained on ImageNet database
\subsubsection{Generating cover and stego datasets}

\subsubsection{Fine tuning the pretrained model}

\subsection{Results}
\subsubsection{Naive insertion}
\subsubsection{Side information}
\subsubsection{Modified cost}

\section*{Conclusion}

\section*{References}
\end{document}